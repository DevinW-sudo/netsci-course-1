\documentclass[11pt,article,oneside]{memoir} %{{{
% based on Kieran Healy's syllabus templates
% https://github.com/kjhealy/latex-custom-kjh

\usepackage{org-preamble-pdflatex}

\setlength{\parskip}{10pt}
\setlength{\parindent}{0pt}

% Definitions
\def\myauthor{Author}
\def\mytitle{Title}
\def\mycopyright{\myauthor}
\def\mykeywords{}
\def\mybibliostyle{plain}
\def\mybibliocommand{}
\def\mysubtitle{}
\def\myaffiliation{Indiana University}
\def\myaddress{Info East Rm 316}
\def\myemail{yyahn@iu.edu}
\def\myweb{http://yongyeol.com}
\def\myphone{856-2920}
\def\myversion{}
\def\myrevision{}

\def\myaffiliation{\ \\Indiana University}
\def\myauthor{Yong-Yeol (YY) Ahn}
\def\mykeywords{Visualization, Data, Undergraduate, Informatics}
\def\mysubtitle{Syllabus}
\def\mytitle{{\normalsize \textsc{Info} I606 \newline} \HUGE Network Science}

%}}}

\begin{document} %{{{

%{{{ chapter style
%%\chapterstyle{article-3}
%\pagestyle{kjh}

\def\ind{\hangindent=1 true cm\hangafter=1 \noindent}
\def\labelitemi{$\cdot$}

\chapterstyle{article-4}  % alternative styles are defined in latex-custom-kjh/needs-memoir/

%}}}
%{{{ Title
\title{\LARGE \mytitle}
\author{\Large\myauthor \newline \footnotesize\texttt{\noindent\myemail}}
\date{Spring 2019 %
\newline TR 9:30am--10:45am
\newline Geology 436 (GY 436)
\newline Office hours: Wed 4pm-5pm or DM me on Slack; on \url{https://iu.zoom.us/my/yyahn}}

%\published{\sffamily I590/H400/I400 / Fall 2014 / Mon \& Wed 4:00--5:15pm / Info West 107 (M) \& 109 (W)}
\maketitle
%}}}
%{{{ TA
\vspace{-20pt}
{\bfseries Assistant Instructor} \\ Elise Jing (\texttt{jingy@iu.edu}); Office Hours: Thursday 11am-Noon
%}}}

\section{Course Description} %{{{

Networks, or graphs, provide a unifying framework to study complex systems, such as living organisms, societies, and many techno-social systems.
This graduate-level course focuses on the fundamental concepts as well as key applications of network science.
The course will cover recent advancement of network science, with respect to statistical properties and models of real-world networks, network algorithms, and practical applications.
Topics include: how information and diseases spread in our society, measures and algorithms for quantifying importance, link prediction, and community detection.

%questions: why do networks matter? What are the fundamental frameworks and
%theories to understand the structure and dynamics of networks?  How has network
%framework been applied to other fields? What are the frontiers of the research?

%}}}
\section{Course Objectives} %{{{

By the end of the course, students are expected to be able to identify, construct, and analyze networks by choosing and applying appropriate methods and algorithms.
Students are also expected to be able to explain, both mathematically and conceptually, the key network concepts and statistical properties, and their implications.
%}}}
\section{Communication} %{{{

We will use Slack as the main communication channel.
So the first thing you need to do is to join the course slack.
The URL for the course Slack site is: \url{https://iu-netsci-course.slack.com}.
You can create an account by using one of the following IU email addresses: \texttt{indiana.edu}, \texttt{umail.iu.edu}, \texttt{iu.edu}, \texttt{iupui.edu}.
If you have any issues joining Slack, please contact the instructor.
Please review the pinned FAQs document on Slack about how to communicate effectively.

%\url{https://yy.github.io/dviz-course/}.
%}}}
\section{Prerequisites} %{{{ \label{sec:Prerequisites}

The course will require a good foundation of mathematics, statistics, and programming,
although there is no formal prerequisite. Key preprequisite topics are:
probability, statistics, linear algebra, data structures, and algorithms.
Python is used as the main programming language and it will be very helpful to
be proficient in Python. Please contact the instructor if you are uncertain
about your background.

%}}}
\section{Requirements and Evaluation} %{{{ \label{sec:requirements}

This course is not driven by the lectures, but by your participation and engagement.
So be prepared to lead your own learning!

Students are required to read assigned readings, attend the classes, complete quizzes and assignments, and engage in (online) discussions.

\textbf{Residential students only}: bring a few markers.

The main evaluation will be based on an exam and a class project.
The project can be conducted individually or by forming a small team.
Students may choose any network-related topics that involve network analysis or modeling, although it is strongly encouraged at least to seek guidance from the instructor.
For more information about the projects, please visit \url{https://github.com/yy/netsci-course/wiki/Projects}.

%}}}
\section{Books and key materials} %{{{

We will closely follow the
\href{http://barabasi.com/networksciencebook/}{Network Science} by Albert-Lásló
Barabási and
\href{https://www.amazon.com/Networks-Mark-Newman/dp/0198805098}{Networks:
An Introduction} by Mark Newman. The following books can be also helpful:

\subsection{Network science} %{{{

\begin{enumerate}
    \item \href{https://www.amazon.com/Networks-Crowds-Markets-Reasoning-Connected/dp/0521195330}{Networks, Crowds, and Markets: Reasoning about a Highly Connected World} by David Easley and Jon Kleinberg.
    \item \href{https://www.amazon.com/Linked-Everything-Connected-Business-Everyday/dp/0465085733}{Linked: How Everything Is Connected to Everything Else and What It Means for Business, Science, and Everyday Life} by Albert-Lásló Barabási.
\end{enumerate}

%}}}
\subsection{Python and data analysis} %{{{

\begin{enumerate}

\item \href{http://www.diveintopython3.net/index.html}{Dive Into Python} by Mark Pilgrim (available online): a good Python book.

\item \href{http://www.learnpython.org}{Learnpython.org}: A web-based interactive tutorial.

\item \href{http://work.thaslwanter.at/Stats/html/}{An introduction to statistics} (with Python) by Thomas Haslwanter (available online): this book uses Python to explain basic statistics. It also contains a succinct tutorial for Python and data visualization using Python.

\item \href{https://www.amazon.com/Mining-Social-Web-Facebook-LinkedIn/dp/1449367615}{Mining the Social Web: Data Mining Facebook, Twitter, LinkedIn, Google+, GitHub, and More} by Matthew A. Russell.

\item \href{http://ipython.rossant.net}{Learning IPython for Interactive Computing and Data Visualization} by  Cyrille Rossant: Introduction to IPython as well as lots of advanced analysis


\end{enumerate} %}}}

%}}}
\section{Policies and advices} %{{{

\begin{enumerate}

\item \emph{Disabilities.} Every attempt will be made to accommodate qualified
students with disabilities (e.g. mental health, learning, chronic health,
physical, hearing, vision, neurological, etc.). You must have established your
eligibility for support services through Disability Services for Students. Note
that services are confidential, may take time to put into place, and are not
retroactive.  Captions and alternate media for print materials may take three
or more weeks to get produced. Please contact Disability Services for Students
at \url{http://disabilityservices.indiana.edu} or 812-855-7578 as soon as
possible if accommodations are needed. The office is located on the third
floor, west tower, of the Wells Library (Room W302). Walk-ins are welcome 8 AM
to 5 PM, Monday through Friday. You can also locate a variety of campus
resources for students and visitors who need assistance at
\url{http://www.iu.edu/~ada/index.shtml}.

\item \emph{Sexual misconduct and Title IX.} As your instructor, one of my
responsibilities is to create a positive learning environment for all students.
Title IX and IU's Sexual Misconduct Policy prohibit sexual misconduct in any
form, including sexual harassment, sexual assault, stalking, and dating and
domestic violence.  If you have experienced sexual misconduct, or know someone
who has, the University can help. If you are seeking help and would like to
speak to someone confidentially, you can make an appointment with:

\begin{enumerate}

\item The Sexual Assault Crisis Services (SACS) at (812) 855-8900 (counseling services)
\item Confidential Victim Advocates (CVA) at (812) 856-2469 (advocacy and advice services)
\item IU Health Center at (812) 855-4011 (health and medical services)

\end{enumerate}

It is also important that you know that Title IX and University policy require
me to share any information brought to my attention about potential sexual
misconduct, with the campus Deputy Title IX Coordinator or IU's Title IX
Coordinator. In that event, those individuals will work to ensure that
appropriate measures are taken and resources are made available. Protecting
student privacy is of utmost concern, and information will only be shared with
those that need to know to ensure the University can respond and assist. I
encourage you to visit \emph{stopsexualviolence.iu.edu} to learn more.

\item \emph{Be honest.} Your assignments and papers should be your own work.
If you find useful resources for your assignments, share them and cite them. If
your friends helped you, acknolwedge them. Feel free to discuss both online and
offline, but you should not show your solution nor see other's. Any cases of
serious academic misconduct (cheating, fabrication, plagiarism, etc) will be
reported to the School and the Dean of Students, following the standard
procedure. But more than anything, cheating will hurt you in a long-term and
\emph{not cool}.

\item \emph{You have the responsibility of backing up all your data and code}.
Always use at least a cloud storage service such as Box, Dropbox, or Google
Drive. Ideally, learn version control systems and use
\url{https://github.iu.edu} or \url{https://github.com}. Loss of data, code, or
papers (e.g. malfunction of your laptop) is not an acceptable excuse.

\item If you have any mental health issues, don't hesistate to contact
\href{http://jhealthcenter.indiana.edu/counseling/index.shtml}{IU's Counseling
and Psychological Services}. They provide free and confidential counseling
sessions.


\end{enumerate}
%}}}
\section{Grading} %{{{
\label{sec:grading_tentative_}

Note that there may be a curve at the end of the class and the grade (percentage) that you see on the Canvas does not necessarily correspond to your final grade.

\begin{itemize}

\item Participation (attendance, quiz, and discussion): 25\% (there will be extra participation credit for answering questions and helping others on slack)

\item Assignments: 30\%

\item Exam: 15\%

\item Project: 30\%

\end{itemize}
%}}}
\section{Course Schedule} %{{{

(The schedule is subject to change)
%Find the most up-to-date schedule at \url{https://github.com/yy/dviz-course/wiki/Schedule}

\subsection{Week 1 (1/8-):  Get ready! Why do we care?}
\vspace{-0.2em}\begin{itemize}\itemsep0em
\item Network Science Ch. 1
\end{itemize}
\subsection{Week 2 (1/15-):  Friendship paradox: a life lesson}
\vspace{-0.2em}\begin{itemize}\itemsep0em
\item Network Science Ch. 2
\item S.L. Feld, ``Why Your Friends Have More Friends Than You Do'', American Journal of Sociology 96, 1464 (1991).
\item N.O. Hodas et al., ``Friendship Paradox Redux: Your Friends Are More Interesting Than You'', ICWSM'13
\item Y.-H. Eom and H.-H. Jo, ``Generalized friendship paradox in complex networks: The case of scientific collaboration'', Scientific Reports 4, 4603 (2014)
\item J.P. Bagrow et al., ``Which friends are more popular than you? Contact strength and the friendship paradox in social networks'', ASONAM'17
\item J. Bollen et al., ``The happiness paradox: your friends are happier than you'', EPJ Data Science 6, 4 (2017).
\end{itemize}
\subsection{Week 3 (1/22-):  ``What a small world!''}
\subsection{Week 4 (1/29-):  Strength of weak ties}
\subsection{Week 5 (2/5-):  Scale-free networks? Steak-pun networks?}
\subsection{Week 6 (2/12-):  Network centralities (Project proposal: 2/12)}
\subsection{Week 7 (2/19-):  Network structure I: communities and other properties}
\subsection{Week 8 (2/26-):  Network structure II: communities and other properties}
\subsection{Week 9 (3/5-):  Theory of random graphs}
\subsection{\color{gray}Week 10 (3/12-): Spring break}
\subsection{Week 11 (3/19-): Network epidemics and robustness}
\subsection{Week 12 (3/26-): Social influence and information diffusion}
\subsection{Week 13 (4/2-): Network embedding}
\subsection{Week 14 (4/9-): Exam week}
\subsection{Week 15 (4/16-): Project hack week}
\subsection{Week 16 (4/23-): Final presentations (Final presentation: 4/23)}
\subsection{Week 17 (4/30-): Final Week. (Project paper: 5/3)}

%}}}

\end{document} %}}}
