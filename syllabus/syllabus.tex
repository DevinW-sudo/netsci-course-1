\documentclass[11pt,article,oneside]{memoir}
% based on Kieran Healy's syllabus templates
% https://github.com/kjhealy/latex-custom-kjh 

\usepackage{org-preamble-pdflatex} 

\setlength{\parskip}{10pt}
\setlength{\parindent}{0pt}

% Definitions
\def\myauthor{Author}
\def\mytitle{Title}
\def\mycopyright{\myauthor}
\def\mykeywords{}
\def\mybibliostyle{plain}
\def\mybibliocommand{}
\def\mysubtitle{}
\def\myaffiliation{Indiana University}
\def\myaddress{Info East Rm 316} 
\def\myemail{yyahn@iu.edu}
\def\myweb{http://yongyeol.com}
\def\myphone{856-2920}
\def\myversion{}
\def\myrevision{}

\def\myaffiliation{\ \\Indiana University}
\def\myauthor{Yong-Yeol (YY) Ahn}
\def\mykeywords{Visualization, Data, Undergraduate, Informatics}
\def\mysubtitle{Syllabus}
\def\mytitle{{\normalsize \textsc{Info} I590 \newline} \HUGE Network Science}

\begin{document}

%%\chapterstyle{article-3}
%\pagestyle{kjh}

\def\ind{\hangindent=1 true cm\hangafter=1 \noindent}
\def\labelitemi{$\cdot$}

\chapterstyle{article-4}  % alternative styles are defined in latex-custom-kjh/needs-memoir/

\title{\LARGE \mytitle}     
\author{\Large\myauthor \newline \footnotesize\texttt{\noindent\myemail}}
\date{Spring 2017 Online. %
%\newline MW 4:00pm--5:15pm. 
\newline Office hours: TBD; anytime on Slack}

%\published{\sffamily I590/H400/I400 / Fall 2014 / Mon \& Wed 4:00--5:15pm / Info West 107 (M) \& 109 (W)}
\maketitle

\vspace{-20pt}
{\bfseries Assistant Instructor} \\ TBD %Nathaniel Rodriguez (\texttt{njrodrig@umail.iu.edu}) \\ Office Hours: Wed 4pm-5pm 

%\section{Course Homepage}

%This syllabus may contain outdated information. For the most accurate and up-to-date information, 
%please check canvas. %\url{https://yy.github.io/dviz-course/}. 

\section{Course Description}

Networks provide a unifying framework to study complex systems and many types
of relational data. This course focuses on the fundamentals and key
applications of network science, addressing the following questions: why do
networks matter? what are the fundamental frameworks and theories to understand
the structure and dynamics of networks? how has network framework been applied
to other fields?  What are the frontiers of the research?

\section{Course Objectives}

By the end of the course, you will be able to identify, construct, visualize,
and analyze networks by choosing and applying appropriate methods and
algorithms. You will be able to both mathematically and conceptually explain
the key concepts and findings of network science. 

\section{Prerequisites}
\label{sec:Prerequisites}

The course will require strong working knowledge of mathematics and
programming. Key prerequisites are: probability, statistics, linear algebra,
data structures, and algorithms. Python is used as the main programming
language.  Contact the instructor if you are uncertain about your background. 

\section{Requirements}
\label{sec:requirements}

Students are required to, not only watch the lectures, but also complete
quizzes and assignments, and engage in online discussions. In addition, each
student will finish a (replication) project. 

\section{Books and key materials}

We will closely follow the
\href{http://barabasi.com/networksciencebook/}{Network Science} by Albert-Lásló
Barabási. The following books can be helpful:

\subsection{Python and data analysis}

\begin{enumerate}

\item \href{http://www.diveintopython3.net/index.html}{Dive Into Python} by Mark Pilgrim (available online): a good Python book. 

\item \href{http://www.learnpython.org}{Learnpython.org}: A web-based interactive tutorial. 

\item \href{http://work.thaslwanter.at/Stats/html/}{An introduction to statistics} (with Python) by Thomas Haslwanter (available online): this book uses Python to explain basic statistics. It also contains a succinct tutorial for Python and data visualization using Python. 

\item \href{http://ipython.rossant.net}{Learning IPython for Interactive Computing and Data Visualization} by  Cyrille Rossant: Introduction to IPython as well as lots of advanced analysis 


\end{enumerate}

\subsection{Network Science}

\begin{enumerate}

\item \href{https://www.amazon.com/Networks-Introduction-Mark-Newman/dp/0199206651}{Networks: An Introduction} by Mark Newman. 

\end{enumerate}

\section{Policies}

\begin{enumerate}

\item \emph{Disabilities.} Every attempt will be made to accommodate qualified
students with disabilities (e.g. mental health, learning, chronic health,
physical, hearing, vision, neurological, etc.). You must have established your
eligibility for support services through Disability Services for Students. Note
that services are confidential, may take time to put into place, and are not
retroactive.  Captions and alternate media for print materials may take three
or more weeks to get produced. Please contact Disability Services for Students
at \url{http://disabilityservices.indiana.edu} or 812-855-7578 as soon as
possible if accommodations are needed. The office is located on the third
floor, west tower, of the Wells Library (Room W302). Walk-ins are welcome 8 AM
to 5 PM, Monday through Friday. You can also locate a variety of campus
resources for students and visitors who need assistance at
\url{http://www.iu.edu/~ada/index.shtml}. 

\item \emph{Be honest.} Your assignments and papers should be your own work.
First, if you find useful resources for your assignments, share them and cite
them. If your friends helped you, acknolwedge them. Second, feel free to
discuss both online and offline, but you should not show your code (papers) nor
see other's. Any cases of academic misconduct (cheating, fabrication,
plagiarism, etc) will be immediately reported to the School and the Dean of
Students, following the standard procedure. Cheating is not cool. 

\item \emph{You have the responsibility of backing up all your data and code}.
Always use at least Box, Dropbox, or Google Drive. Ideally, learn version
control systems and use \url{https://github.iu.edu} or
\url{https://github.com}. Loss of data, code, or papers due to various reasons
(e.g. malfunction of your laptop) is not an acceptable excuse for delayed or
missing submission. 

\item If you have any issues, don't hesistate to contact me or
\href{http://healthcenter.indiana.edu/counseling/index.shtml}{IU's Counseling
and Psychological Services}. 


\end{enumerate}

\section{Grading (Tentative)}
\label{sec:grading_tentative_}

\begin{itemize}

\item Participation (quizzes and discussions): 30\%

\item Assignments: 50\%

\item Project: 20\%


\end{itemize}

\section{Course Schedule}

(The schedule is subject to change)
%Find the most up-to-date schedule at \url{https://github.com/yy/dviz-course/wiki/Schedule}

\subsection{Week 1:  Why Networks?}
\subsection{Week 2:  Graphs and Small World}
\subsection{Week 3:  Weak Ties and Watts-Strogatz Model}
\subsection{Week 4:  Scale-free Networks}
\subsection{Week 5:  Power-law or Not?}
\subsection{Week 6:  Centralities} 
\subsection{Week 7:  Communities I}
\subsection{Week 8:  Communities II}
\subsection{Week 9:  Theory of Random Graphs}
\subsection{Week 10: Random Walks}
\subsection{Week 11: Biological Networks}
\subsection{Week 12: Epidemics}
\subsection{Week 13: Robustness}
\subsection{Week 14: Thanksgiving}
\subsection{Week 15: Information Diffusion I}
\subsection{Week 16: Information Diffusion II}

\end{document}
